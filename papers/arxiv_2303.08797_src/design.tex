In this section, we study the stochastic interpolants that are obtained when we specialize the function $I$ used in~\eqref{eq:stochinterp} to be linear in both $x_0$ and $x_1$, i.e. we consider
\begin{equation}
    \label{eq:lin:interp}
    x^\LIN_t = \alpha(t) x_0+ \beta(t) x_1 + \gamma(t) z,
\end{equation} 
where $(x_0,x_1)\sim \nu$ and $z \sim {\sf N}(0,\Id)$ with $(x_0,x_1)\perp z$, and $\alpha, \beta, \gamma^2\in C^2([0,1])$ satisfy the conditions
\begin{equation}
    \label{eq:lin:interp:a:b}
    \begin{aligned}
         & \alpha(0)=\beta(1)=1; \quad \alpha(1)=\beta(0)=\gamma(0) = \gamma(1) = 0;  \quad \forall t\in(0,1) \ : \ \gamma(t) > 0.
    \end{aligned}
\end{equation}
Despite its simplicity, this setup offers significant design flexibility. The discussion highlights how the presence of the latent variable $\gamma(t)z$ can simplify the structure of the intermediate density $\rho(t)$. 
%
Since our ultimate aim is to investigate the properties of practical generative models built upon ODEs or SDEs, we will also study the effect of time-dependent diffusion coefficient $\eps(t)$, which controls the amplitude of the noise in a generative SDE. 
%
Throughout, to build intuition, we choose $\rho_0$ and $\rho_1$ to be Gaussian mixture densities, for which the drift coefficients can be computed analytically (see Appendix~\ref{app:Gauss:mixt}). This enables us to visualize the effect of each choice on the resulting generative models.


\subsection{Factorization of the velocity field}
\label{sec:factor} 

When the stochastic interpolant is of the form~\eqref{eq:lin:interp}, the velocity~$b$ and the score~$s$ defined in~\eqref{eq:b:ode:def} and \eqref{eq:s:def} can both be expressed in terms of the following three conditional expectations (the third is the denoiser already defined in~\eqref{eq:denoiser}):
\begin{equation}
    \label{eq:g:c}
    \eta_0(t,x) = \EE(x_0|x^\LIN_t = x), \quad \eta_1(t,x) = \EE(x_1|x^\LIN_t = x), \quad \eta_z(t,x) = \EE(z|x^\LIN_t = x).
\end{equation}
Specifically, we have
\begin{equation}
    \label{eq:b:s:g}
    b(t,x) = \dot\alpha(t) \eta_0(t,x) +  \dot\beta(t) \eta_1(t,x)+  \dot\gamma(t) \eta_z(t,x), \qquad s(t,x) = - \gamma^{-1}(t) \eta_z(t,x).
\end{equation}
The second relation above holds for $t\in(0,1)$ (i.e. when $\gamma(t) \not=0$).
Moreover, because $\EE (x_t|x_t=x) = x$ by definition, the functions $\eta_0$, $\eta_1$, and $\eta_z$ satisfy
\begin{equation}
    \label{eq:constraint}
    \forall (t,x) \in [0,1]\times \RR^d \quad : \quad \alpha(t) \eta_0(t,x) + \beta(t) \eta_1(t,x) + \gamma(t) \eta_z(t,x) = x.
\end{equation}
This enables us to reduce computational expense: given two of the $\eta$'s, the third can always be calculated via~\eqref{eq:constraint}.
 Finally, it is easy to see that  the functions $\eta_0$, $\eta_1$, and $\eta_z$ are the unique minimizers of the objectives
\begin{equation}
    \label{eq:obj}
    \begin{aligned}
        \mathcal L_{\eta_0}(\hat \eta_0) &= \int_0^1 \EE \left[\tfrac12|\hat \eta_0(t,x^\LIN_t)|^2 - x_0\cdot \hat \eta_0(t,x^\LIN_t)\right] dt,\\
        \mathcal L_{\eta_1}(\hat \eta_1) &= \int_0^1 \EE \left[\tfrac12|\hat \eta_1(t,x^\LIN_t)|^2 - x_1\cdot \hat \eta_1(t,x^\LIN_t)\right] dt,\\
        \mathcal L_{\eta_z}(\hat \eta_z) &= \int_0^1 \EE \left[\tfrac12|\hat \eta_z(t,x^\LIN_t)|^2 - z\cdot \hat \eta_z(t,x^\LIN_t)\right] dt,
    \end{aligned}
\end{equation}
where $x^\LIN_t$ is defined in~\eqref{eq:lin:interp} and the expectation is taken independently over $(x_0,x_1)\sim\nu$ and $z\sim {\sf N}(0,\text{\it Id}).$

\begin{table}[t!]
\centering
\renewcommand{\arraystretch}{1.5}  % Adjusts the height of the rows
\begin{tabular}{@{}ccccc@{}}
\toprule
\multicolumn{1}{c}{Stochastic Interpolant}                    &                & $\alpha(t)$              & $\beta(t)$               & $\gamma(t)$      \\ \midrule

\multicolumn{1}{l}{\multirow{3}{*}{Arbitrary $\rho_0$ (two-sided)}} & linear  & $1-t$                    & $t$                    & $\sqrt{at(1-t)}$ \\ \cmidrule(l){2-5}
                                                                     & trig    & $\cos{\tfrac{\pi}{2} t}$ & $\sin{\tfrac{\pi}{2} t}$ & $\sqrt{at(1-t)}$ \\ \cmidrule(l){2-5}
                                                                     & enc-dec &     $\cos^2(\pi t) 1_{[0,\frac12)}(t) $                    &    $\cos^2(\pi t) 1_{(\frac12,1]}(t)   $                   &   $\sin^2(\pi t)$               \\ 
\midrule
\multicolumn{1}{l}{\multirow{3}{*}{Gaussian $\rho_0$ (one-sided)}}  & linear  & $1-t$                    & $t$                      & 0                \\ \cmidrule(l){2-5}
                                                                     & trig    & $\cos{\tfrac{\pi}{2} t}$ & $\sin{\tfrac{\pi}{2} t}$ & 0\\ \cmidrule(l){2-5}
                                                                     & {\small SBDM (VP)}    & $\sqrt{1-t^2}$ & $t$ & 0                \\ 
\midrule
\multicolumn{1}{c}{Mirror}                    &                      & 0                        & 1                        & $\sqrt{at(1-t)}$ \\ \bottomrule
\end{tabular}
\caption{\textbf{\new{Spatially linear interpolants.}} A table suggesting various linear interpolants. 
%
In general, this paper describes methods for arbitrary $\rho_0$ and $\rho_1$.
%
In Section \ref{sec:spatil:lin:os}, we detail linear interpolants for one-sided generation, where $\rho_0$ is a Gaussian and the latent variable $z$ can be absorbed into $x_0$. 
%
Later, in Section~\ref{sec:SBDM}, we discuss how to recast score-based diffusion models (SBDM) as linear one-sided interpolants, which leads to the expressions given in the table when using a variance-preserving parameterization.
%
Linear mirror interpolants, where $\rho_0 = \rho_1$ are equal, are defined by~\eqref{eq:stochinterp:mirror} with the choice $K(t,x_1) = \alpha(t)x_1$.}
\end{table}

\subsection{Some specific design choices}
\label{sec:specific:a:b:c}
It is useful to assume that both $\rho_0$ and $\rho_1$ have been scaled to have zero mean and identity covariance (which can be achieved in practice, for example, by an affine transformation of the data). In this case, the time-dependent mean and covariance of~\eqref{eq:lin:interp} are given by
\begin{equation}
\label{eq:mean:var}
\EE [x^\LIN_t] = 0, \qquad \EE [ x^\LIN_t (x^\LIN_t)^\T] = (\alpha^2(t)+\beta^2(t)+\gamma^2(t)) \Id.
\end{equation}
Preserving the identity covariance at all times therefore leads to the constraint
\begin{equation}
\label{eq:sumto1}
\forall t \in [0,1] \quad : \quad \alpha^2(t)+\beta^2(t)+\gamma^2(t) = 1.
\end{equation}
%
This choice is also sensible if $\rho_0$ and $\rho_1$ have covariances that are not the identity but are on a similar scale. In this case we no longer need to enforce~\eqref{eq:sumto1} exactly, and could, for example, take three functions whose sum of squares is of order one. For definiteness, in the sequel we discuss possible choices that satisfy~\eqref{eq:sumto1} exactly, with the understanding that the corresponding functions $\alpha$, $\beta$, and $\gamma$ could all be slightly modified without significantly affecting the conclusions.

\paragraph{Linear and trigonometric $\alpha$ and $\beta$.} 
%
One way to ensure that~\eqref{eq:sumto1} holds while maintaining the influence of $\rho_0$ and $\rho_1$ everywhere on $[0,1]$ except at the endpoints is to choose 
\begin{equation}
\label{eq:lin:a:b:c}
\alpha(t) = t, \qquad \beta(t) = 1-t, \qquad \gamma(t) = \sqrt{2t(1-t)}.
\end{equation}
This choice was advocated in~\cite{liu2022}, without the inclusion of the latent variable ($\gamma =0$). Another possibility that gives more leeway is to pick any $\gamma: [0,1]\to[0,1]$ and set
\begin{equation}
\label{eq:trig}
\alpha(t) = \sqrt{1-\gamma^2(t)} \cos(\tfrac 12 \pi t), \qquad \beta(t) = \sqrt{1-\gamma^2(t)} \sin(\tfrac 12 \pi t).
\end{equation}
With $\gamma=0$, this was the choice preferred in~\cite{albergo2023building}. The PDF $\rho(t)$ obtained with the choices~\eqref{eq:lin:a:b:c} and~\eqref{eq:trig} when $\rho_0$ and $\rho_1$ are both Gaussian mixture densities are shown in Figure~\ref{fig:gmm_rhot}. As this example shows, when $\rho_0$ and $\rho_1$ have distinct complex features, these would be duplicated in $\rho(t)$ at intermediate times if not for the smoothing effect of the latent variable; this behavior is seen in Figure~\ref{fig:gmm_rhot}, where it is most prominent in the first row with $\gamma(t) = 0$. From a statistical learning perspective, eliminating the formation of spurious features will simplify the estimation of the velocity field $b$, which becomes smoother as the formation of such features is suppressed.



\label{sec:reg:noise}
\begin{figure}[t!]
    \centering
    \includegraphics[width=\textwidth]{figs/gmm_rhot.pdf}   
    \caption{\textbf{The effect of $\gamma(t)z$ on $\rho(t)$.} A visualization of how the choice of $\gamma(t)$ changes the density $\rho(t)$ of $x^\LIN_t=\alpha(t) x_0 + \beta(t) x_1+ \gamma(t)z $ when $\rho_0$ and $\rho_1$ are Gaussian mixture densities with two modes and three modes, respectively.
    %
    The first row depicts $\gamma(t)=0$, which reduces to the stochastic interpolant developed in~\cite{albergo2023building}. This case forms a valid transport between $\rho_0$ and $\rho_1$, but produces spurious intermediate modes in $\rho(t)$.
    %
    The second row depicts the choice of $\gamma(t) = \sqrt{2t(1-t)}$. In this case, the spurious modes are partially damped by the addition of the latent variable, leading to a simpler $\rho(t)$.
    %
    The final row shows the Gaussian encoding-decoding, which smoothly encodes $\rho_0$ into a standard normal distribution on the time interval $[0, 1/2)$, which is then decoded into $\rho_1$ on the interval $(1/2, 1]$. In this case, no intermediate modes form in $\rho(t)$: the two modes in $\rho_0$ collide to form $\mathsf{N}(0, 1)$ at $t=\tfrac12$, which then spreads into the three modes of $\rho_1$. 
    % 
    A visualization of individual sample trajectories from deterministic and stochastic generative models based on ODEs and SDEs whose solutions have density $\rho(t)$ can be seen in Figure~\ref{fig:gmm_trajs}.
    }
    \label{fig:gmm_rhot}
\end{figure}


\begin{figure}[t!]
    \centering
    \includegraphics[width=\textwidth]{figs/gmm_trajs.pdf}   
    \caption{\textbf{The effect of $\epsilon$ on sample trajectories.} A visualization of how the choice of $\epsilon$ affects the sample trajectories obtained by solving the ODE~\eqref{eq:ode:1} or the forward SDE~\eqref{eq:sde:1}. The set-up is the same as in Figure~\ref{fig:gmm_rhot}:
    %
    $\rho_0$ and $\rho_1$ are taken to be the same Gaussian mixture densities, and the analytical expressions for $b$ and $s$ are used.
    %
    In the three panels in each column the value of $\gamma$ is the same, and each panel shows trajectories with different $\eps$. 
    Three specific trajectories from the same three initial conditions drawn from $\rho_0$ are also highlighted in white in every panel.
    %
    As $\epsilon$ increases but $\gamma$ stays the same, the density $\rho(t)$ is unchanged, but the individual trajectories become increasingly stochastic.
    %
    While all choices are equivalent with exact $b$ and $s$, Theorem~\ref{thm:kl:bound} shows that nonzero values of $\epsilon$ provide control on the likelihood in terms of the error in $b$ and $s$ when they are approximate.}
    \label{fig:gmm_trajs}
\end{figure}

\paragraph{Gaussian encoding-decoding.}
A useful limiting case is to devolve the data from $\rho_0$ completely into noise by the halfway point $t=\tfrac12$ and to reconstruct $\rho_1$ completely from noise starting from $t=\frac12$. One choice that allows us to do so while satisfying~\eqref{eq:sumto1} is
\begin{equation}
    \label{eq:alpha:beta:2}
    \alpha(t) = \cos^2(\pi t) 1_{[0,\frac12)}(t), \qquad \beta(t) = \cos^2(\pi t)1_{(\frac12,1]}(t), \qquad \gamma(t) = \sin^2(\pi t),
\end{equation}
where $1_{A}(t)$ is the indicator function of $A$, i.e. $1_A(t) =1$ if $t\in A$ and $1_A(t)=0$ otherwise.
With this choice, it is easy to see that $x_{t=\frac12} = \gamma(\tfrac{1}{2}) z \sim {\sf N}(0,\gamma^2(\tfrac{1}{2}))$, which seamlessly glues together two interpolants: one between $\rho_0$ and a standard Gaussian, and one between a standard Gaussian and $\rho_1$.

Even though the choice~\eqref{eq:alpha:beta:2} encodes $\rho_0$ into pure noise on the interval $[0,\tfrac12]$, which is then decoded into $\rho_1$ on the interval $[\tfrac12,1]$ (and vice-versa when proceeding backwards in time), the resulting velocity $b$ still defines a single continuity equation that maps $\rho_0$ to $\rho_1$ on $[0, 1]$. 
%
This is most clearly seen at the level of the probability flow~\eqref{eq:ode:1}, since its solution $X_t$ is a bijection between the initial and final conditions $X_{t=0}$ and $X_{t=1}$, but a similar pairing can also be observed in the solutions to the forward and backward SDEs~\eqref{eq:sde:1} and~\eqref{eq:sde:R}, whose solutions at time $t=1$ or $t=0$ remain correlated with the initial or final condition used. 
%
\begin{wrapfigure}[10]{r}{0.6\textwidth}
\centering
\vspace{-0.0cm}
  \includegraphics[width=1.0\linewidth]{figs/alphas-betas-gammas.pdf}
  \caption{\textbf{The functions $\alpha(t)$, $\beta(t)$, and $\gamma(t)$} for the linear \eqref{eq:lin:a:b:c}, trigonometric~\eqref{eq:trig} with $\gamma(t)=\sqrt{2t(1-t)}$, Gaussian encoding-decoding~\eqref{eq:alpha:beta:2}, one-sided~\eqref{eq:stochinterp:os}, and mirror~\eqref{eq:stochinterp:mirror} interpolants.}
  \label{fig:interps}
\end{wrapfigure}
%
This allows for a more direct means of image-to-image translation with diffusions when compared to the recent approach described in \cite{su2023dual}. 
%
The choice~\eqref{eq:alpha:beta:2} is depicted in the final row of Figure~\ref{fig:gmm_rhot}, where no spurious modes form at all; individual sample trajectories of the deterministic and stochastic generative models based on ODEs and SDEs whose solutions have this $\rho(t)$ as density can be seen in the panels forming the third column in Figure~\ref{fig:gmm_trajs}.
%
We note that the elimination of spurious intermediate modes can also be implemented by use of a data-adapted coupling $\nu(dx_0, dx_1)$, as considered in~\cite{albergo_stochastic_2023}.

Unsurprisingly, it is necessary to have $\gamma(t) > 0$ for the choice~\eqref{eq:alpha:beta:2}: for $\gamma(t) = 0$, the density $\rho(t)$ collapses to a Dirac measure at $t=\frac12$. This consideration highlights that the inclusion of the latent variable $\gamma(t) z$ matters even for the deterministic dynamics~\eqref{eq:ode:1}, and its presence is distinct from the stochasticity inherent to the SDEs~\eqref{eq:sde:1} and \eqref{eq:sde:R}.

\subsection{Impact of the latent variable $\gamma(t) z$ and the diffusion coefficient~$\epsilon(t)$}
\label{sec:impact:gam}

%
The stochastic interpolant framework enables us to discern the independent roles of the latent variable $\gamma(t) z$ and the diffusion coefficient $\epsilon(t)$ we use in a generative model.
%
As shown in Theorem~\ref{prop:interpolate}, the presence of the latent variable $\gamma(t) z$ for $\gamma \neq 0$ smooths both the density $\rho(t)$ and the velocity $b$ defined in~\eqref{eq:b:ode:def} spatially.
%
This provides a computational advantage at sample generation time because it simplifies the required numerical integration of~\eqref{eq:ode:1},~\eqref{eq:sde:1}, and~\eqref{eq:sde:R}. 
%
Intuitively, this is because the density $\rho(t)$ of $x_t$ can be represented exactly as the density that would be obtained with $\gamma(t) = 0$ convolved with $\mathsf{N}(0, \gamma^2(t)Id)$ at each $t\in(0,1)$. A comparison between the density $\rho(t)$ obtained with trigonometric interpolants with $\gamma(t) = 0$ and $\gamma(t) = \sqrt{2t(1-t)}$ can be seen in the first and second row of Figure~\ref{fig:gmm_rhot}.

By contrast, the diffusion coefficient $\eps(t)$ leaves the density  $\rho(t)$ unchanged, and only affects the way we sample it. 
%
In particular, the probability flow ODE~\eqref{eq:ode:1} results in a map that pushes every $X_{t=0}=x_0$ onto a single $X_{t=1}=x_1$ and vice-versa. 
%
The forward SDE~\eqref{eq:sde:1} maps each $X^\fwd_{t=0}=x_0$ onto an ensemble $X^\fwd_{t=1} $ whose spread is controlled by the amplitude of $\epsilon(t)$ (and similarly for the reversed SDE ~\eqref{eq:sde:R} that maps each $X^\rev_{t=1}=x_1$ onto an ensemble $X^\rev_{t=0}$). 
%
This ensemble is not distributed according to $\rho_1$ for finite $\epsilon(t)$ -- like with the ODE, we need to sample initial conditions from $\rho_0$ to get solutions at time $t=1$ that sample $\rho_1$ -- but its density converges towards $\rho_1$ as $\epsilon(t) \to \infty$. 
%
These features are illustrated in Figure~\ref{fig:gmm_trajs}. 

\begin{remark}
\label{rem:endpoints}
    Another potential advantage of including the latent variable $\gamma(t) z$ is its impact on the velocity $b$ at the end points. Since $x_{t=0}=x_0$ and $x_{t=1}=x_1$, it is easy to see that the velocity $b$ of the linear interpolant $x_t$ defined in~\eqref{eq:lin:interp} satisfies
\begin{equation}
    \label{eq:vel:0:1}
    \begin{aligned}
    b(0,x) &= \dot\alpha(0) x + \dot \beta(0) \EE [x_1|x_0=x] - \lim_{t\to0} \gamma(t) \dot\gamma(t) s_0(x),\\
    b(1,x) &= \dot\alpha(1) \EE [x_0|x_1=x] + \dot \beta(1) x - \lim_{t\to1} \gamma(t) \dot\gamma(t) s_1(x),
    \end{aligned}
\end{equation}
where  $s_0 = \nabla \log \rho_0$ and $s_1 = \nabla \log \rho_1$. If $\gamma\in C^2([0,1])$, because $\gamma(0)=\gamma(1)=0$, the terms involving the scores $s_0$ and $s_1$ in these expressions vanish. Choosing $\gamma^2 \in C^1([0,1])$ but $\gamma$ not differentiable at $t=0$ or $t=1$ leaves open the possibility that the limits remain nonzero. For example, if we take one of the choices discussed in Section~\ref{sec:sisb}, i.e. 
\begin{equation}
    \label{eq:gam:Bt}
    \gamma(t) = \sqrt{a t(1-t)}, \qquad a>0,
\end{equation}
we obtain
\begin{equation}
    \label{eq:gam:Bt:lim}
    \lim_{t\to0}\gamma(t)\dot{\gamma}(t) = - \lim_{t\to1}\gamma(t)\dot{\gamma}(t) = \frac{a}{2}.
\end{equation}
As a result, the choice~\eqref{eq:gam:Bt} ensures that the velocity $b$ encodes information about the score of the densities $\rho_0$ and $\rho_1$ at the end points. 
%
We stress however that, while the choice of $\gamma(t)$ given in \eqref{eq:gam:Bt} is appealing because of its nontrivial influence on the velocity $b$ at the endpoints, the user is free to explore a variety of alternatives. 
%
We present some examples in Table~\ref{tab:gammas}, specifying the differentiability of $\gamma$ at $t=0$ and $t=1$. The function $\gamma(t)$ specified in~\eqref{eq:gam:Bt} is the only featured case for which the contribution from the score is non-vanishing in the velocity $b$ at the endpoints. In Section~\ref{sec:numerics}, we illustrate on numerical examples that there are tradeoffs between different choices of $\gamma$, which might be directly related to this fact. 
%
When using the ODE as a generative model, the score is only felt through $b$, whereas it is explicit when using the SDE as a generative model.
\end{remark} 

\begin{table}[t!]
    \centering
    \captionsetup{justification=raggedright} % Center caption
    \newcolumntype{C}{>{\centering\arraybackslash}p{1.5cm}} % Define a new centered column type with a fixed width
    \begin{tabular}{ccccc}
    \toprule
       $\gamma(t):$ & $\sqrt{a t(1-t)}$ & $ t(1-t)$ & $\hat \sigma(t)$ & $\sin^2(\pi t)$ \\
    \midrule
    $C^1$ at $t=0,1$ & \ding{55}   & \ding{51}   & \ding{51} &  \ding{51} \\
    \bottomrule
    \end{tabular}
    \caption{\textbf{Differentiability of $\gamma(t)z$.} 
    %
    A characterization of the possible choices of $\gamma(t)$ with respect to their differentiability. The column specified by $\hat \sigma(t)$ is sum of sigmoid functions, made compact by the notation $\hat\sigma(t) = \sigma(f (t - \tfrac{1}{2}) + 1) - \sigma(f(t - \tfrac{1}{2}) - 1) - \sigma(-\tfrac{f}{2}+1) + \sigma(-\tfrac{f}{2}-1) $, where $\sigma(t) = e^t/(1+e^t)$ and $f$ is a scaling factor.}
    \label{tab:gammas}
\end{table}


\subsection{Spatially linear one-sided interpolants}
\label{sec:spatil:lin:os}

Much of the discussion above generalizes to one-sided interpolants if we take the function $J(t,x_1)$ in \eqref{eq:stochinterp:os} to be linear in $x_1$ and define
\begin{equation}
    \label{eq:interp:os:lin}
    x^\OSLIN_t = \alpha(t) z + \beta(t) x_1, \qquad t\in[0,1]
\end{equation}
where $\alpha^2,\beta\in C^2([0,1])$ and $\alpha(0)=\beta(1) = 1$, $\alpha(1) = \beta(0) = 0$, and $\alpha(t) >0$ for all $t\in[0,1)$. 
The velocity~$b$ and the score~$s$ defined in~\eqref{eq:b:ode:def} and \eqref{eq:s:def} can now be expressed as
\begin{equation}
    \label{eq:b:ode:os:lin}
    b_\ODE(t,x) = \dot\alpha(t) \eta^\OS_z(t,x) + \dot \beta(t) \eta^\OS_1(t,x), \qquad s(t,x) = -\alpha^{-1}(t) \eta^\OS_z(t,x),
\end{equation}
where the second expression holds for all $t\in [0,1)$ and we defined:
\begin{equation}
    \label{eq:eta:os}
    \eta^\OS_z(t,x) = \EE(z|x^\OSLIN_t = x), \qquad \eta^\OS_1(t,x) = \EE(x_1|x^\OSLIN_t = x).
\end{equation}
Note that, by definition of the conditional expectation, $\eta^\OS_z$ and $\eta^\OS_1$ satisfy
\begin{equation}
    \label{eq:eta:os:c}
    \forall (t,x) \in [0,1]\times \RR^d \quad : \quad \alpha(t) \eta^\OS_z(t,x) + \beta(t) \eta^\OS_1(t,x)  = x. 
\end{equation}
As a result, only one of them needs to be estimated. 
For example, we can express $\eta^\OS_1$ as a function of $\eta^\OS_z$ for all $t$ such that $\beta(t)\not=0$, and use the result to express the velocity~\eqref{eq:b:ode:os:lin} as
\begin{equation}
    \label{eq:b:os:solved}
    b_\ODE(t,x) = \dot \beta(t) \beta^{-1}(t) x + \big( \dot \alpha(t) - \alpha(t) \dot\beta(t) \beta^{-1}(t)\big) \eta^\OS_z(t,x) \qquad \forall t \ : \ \beta(t)\not=0.
\end{equation}
Assuming that $\beta(t)\not=0$ for all $t\in(0,1]$, this formula only needs to be supplemented at $t=0$ with
\begin{equation}
    \label{eq:b:lin:t1}
    b_\ODE(0,x) = \dot\alpha(0) x + \dot\beta(0) \EE[x_1]
\end{equation}
which follows from~\eqref{eq:b:ode:os:lin} since $x^\OSLIN_{t=0} = z$. 
%
Later in Section \ref{sec:denoiser} we will show that using the velocity $b$ in \eqref{eq:b:os:solved} to solve the probability flow ODE \eqref{eq:ode:1} can be seen as using a denoiser to construct a generative model. 

Finally note that $\eta_z$ and/or $\eta_1$ can be estimated using the following two objective functions, respectively:
\begin{equation}
    \label{eq:obj:sbdm:eta}
    \begin{aligned}
    \mathcal L_{\eta_z}(\hat \eta^\OS_z) &= \int_0^1 \EE \left[\tfrac12|\hat \eta^\OS_z(t,x^\OSLIN_t)|^2 - z\cdot \hat \eta^\OS_z(t,x^\OSLIN_t)\right] dt,\\
    \mathcal L_{\eta_1}(\hat \eta^\OS_1) &= \int_0^1 \EE \left[\tfrac12|\hat \eta^\OS_1(t,x^\OSLIN_t)|^2 - x_1\cdot \hat \eta^\OS_1(t,x^\OSLIN_t)\right] dt.
    \end{aligned}
\end{equation}

