\subsection{Background and motivation}
\label{sec:back:mot}

Dynamical approaches for deterministic and stochastic transport have become a central theme in contemporary generative modeling research. At the heart of progress is the idea to use ordinary or stochastic differential equations (ODEs/SDEs) to continuously transform samples from a base probability density function (PDF)~$\rho_0$ into samples from a target PDF~$\rho_1$ (or vice-versa), and the realization that inference over the velocity field in these equations can be formulated as an empirical risk minimization problem over a parametric class of functions \citep{grathwohl2018scalable,Song2019,ho2020,song2021scorebased,benhamu2022,albergo2023building,liu2022,lipman2022}.

A major milestone was the introduction of score-based diffusion methods (SBDM) \citep{song2021scorebased}, which map an arbitrary density into a standard Gaussian by passing samples through an Ornstein-Uhlenbeck (OU) process. The key insight of SBDM is that this process can be reversed by introducing a backwards SDE whose drift coefficient depends on the score of the time-dependent density of the process. By learning this score -- which can be done by minimization of a quadratic objective function known as the denoising loss~\citep{vincent_connection_2011} -- the backwards SDE can be used as a generative model that maps Gaussian noise into data from the target. Though theoretically exact, the mapping takes infinite time in both directions, and hence must be truncated in practice.

While diffusion-based methods have become state-of-the-art for tasks such as image generation, there remains considerable interest in developing methods that bridge two \textit{arbitrary} densities (rather than requiring one to be Gaussian), that accomplish the transport \textit{exactly}, and that do so on a \textit{finite} time interval.
%
Moreover, while the highest quality results from score-based diffusion were originally obtained using SDEs~\citep{song2021scorebased}, this has been challenged by recent works that find equivalent or better performance with ODE-based methods if the score is learned sufficiently well~\citep{Karras2022edm}.
%
If made to match the performance of their stochastic counterparts, ODE-based methods exhibit a number of desirable characteristics, such as an exact, computationally tractable formula for the likelihood and the easy application of well-developed adaptive integration schemes for sampling. 
%
It is an open question of significant practical importance to understand if there exists a separation in sample quality between generative models based on deterministic dynamics and those based on stochastic dynamics. 

In order to satisfy the desirable characteristics outlined in the previous paragraph, we develop a framework for generative modeling based on the method proposed in~\cite{albergo2023building}, which is built on the notion of a \textit{stochastic interpolant}~$x_t$ used to bridge  two arbitrary densities $\rho_0$ and $\rho_1$.
%
We will consider more general designs below, but as one example the reader can keep in mind:
\begin{equation}
    \label{eq:stoch:interp:lin}
    x_t = (1-t) x_0 + t x_1 + \sqrt{2t(1-t)} z, \quad t \in [0,1],
\end{equation}
where $x_0$, $x_1$, and $z$ are random variables drawn independently from $\rho_0$, $\rho_1$, and the standard Gaussian density $\mathsf{N}(0,\Id)$, respectively. The stochastic interpolant~$x_t$ defined in~\eqref{eq:stoch:interp:lin} is a continuous-time stochastic process that, by construction, satisfies $x_{t=0} = x_0\sim \rho_0$ and $x_{t=1} = x_1\sim \rho_1$. Its paths therefore \textit{exactly} bridge between samples  from $\rho_0$ at $t=0$ and from $\rho_1$ at $t=1$. A key observation is that:
\begin{quote}
    \textit{The law of the interpolant $x_t$ at any time $t\in[0,1]$ can be realized by many different processes, including an ODE and forward and backward SDEs whose drifts can be learned from data.}
\end{quote} 
To see why this is the case, one must consider the probability distribution of the interpolant~$x_t$. As shown below, for a large class of densities $\rho_0$ and $\rho_1$ supported on $\RR^d$, this distribution is absolutely continuous with respect to the Lebesgue measure. Moreover, its time-dependent density~$\rho(t)$ satisfies a first-order transport equation and a family of forward and backward Fokker-Planck equations in which the diffusion coefficient can be varied at will. Out of these equations, we can readily derive generative models that satisfy ODEs and SDEs, respectively, and whose densities at time~$t$ are given by $\rho(t)$.

\begin{figure}[t]
    \centering
    \includegraphics[width=\linewidth]{figs/front_figure-final.pdf}
    \caption{\textbf{The stochastic interpolant paradigm.} 
    %
    Example generative models based on the proposed framework, which connects two densities $\rho_0$ and $\rho_1$ using samples from both.
    %
    The design of the time-dependent probability density $\rho(t)$ that bridges between $\rho_0$ and $\rho_1$ is separated from the choice of how to sample it, which can be accomplished with deterministic or stochastic generative models. 
    %
    \textit{Left panel:} Sampling with a deterministic (ODE) generative model known as the probability flow equation. 
    %
    \textit{Right panel:} Sampling with a stochastic generative model given by an SDE with a tunable diffusion coefficient. 
    %
    The probability flow equation and the SDE have different paths, but their time-dependent density $\rho(t)$ is the same.
    %
    Moreover, the two equations rely on the same estimates for the velocity and the score. }
    \label{fig:my_label}
\end{figure}

Interestingly, the drift coefficients entering these ODEs/SDEs are the unique minimizers of quadratic objective functions that can be  estimated empirically using data from $\rho_0$, $\rho_1$, and $\mathsf{N}(0,\Id)$. The resulting least-squares regression problem allows us to estimate the drift coefficients of the ODE/SDEs, which can then be used to push samples from $\rho_0$ onto new samples from $\rho_1$ and vice-versa. 

\subsection{Main contributions and organization}
The approach introduced here is a versatile way to build generative models that unifies and extends many existing algorithms. In Sec.~\ref{sec:theo}, we develop the framework in full generality, where we emphasize the following key contributions:
\begin{itemize}[leftmargin=0.2in]
    \item We prove that the stochastic interpolant defined in Section~\ref{sec:si:gm} has a distribution that is absolutely continuous with respect to the Lebesgue measure on $\RR^d$, and that its density $\rho(t)$ satisfies a first-order transport equation (TE) as well as a family of forward and backward Fokker-Planck equations (FPEs) with tunable diffusion coefficients.
    %
    \item We show how the stochastic interpolant can be used to learn the drift coefficients that enter the TE and the FPEs. 
    %
    We characterize these coefficients as the minimizers of simple quadratic objective functions given in Section~\ref{sec:cont:eq}. 
    %
    We introduce a new objective for the score $\nabla\log\rho(t)$ of the interpolant density, as well as an objective function for learning a denoiser $\eta_z$, which we relate to the score.
    %
    \item In Section~\ref{sec:generative}, we derive ordinary and stochastic differential equations associated with the TE and FPEs that lead to deterministic and stochastic generative models.
    %
    In Section~\ref{sec:likelihood_bounds}, we show that regressing the drift for SDE-based models controls the likelihood, but that regressing the drift alone is not sufficient for ODE-based models, which must also minimize a Fisher divergence. We show how to optimally tune the diffusion coefficient to maximize the likelihood for SDEs.
    %
    \item In Section~\ref{sec:density}, we develop a general formula to evaluate the likelihood of SDE-based generative models that serves as a natural counterpart to the continuous change-of-variables formula commonly used to compute the likelihood of ODE-based models. In addition, we give formulas to estimate the cross-entropy.
\end{itemize}

In Section~\ref{sec:generalization}, we discuss  instantiations of the stochastic interpolant method. 
%
In Section~\ref{sec:sb} we first show that interpolants are equivalent to a class of stochastic bridges, but that they avoid the need for Doob's $h$-transform, which is generically unknown; we show that this simplifies the construction of a broad class of generative models.
%
In Section~\ref{sec:onesided}, we define the \textit{one-sided interpolant}, which corresponds to the conventional setting in which the base $\rho_0$ is taken to be a Gaussian.
%
With a Gaussian base, several aspects of the interpolant simplify, and we detail the corresponding objective functions.
%
In Section~\ref{sec:mirror}, we introduce a \textit{mirror interpolant} in which the base $\rho_0$ and the target $\rho_1$ are identical.
%
Finally, in Section~\ref{sec:sb}, we show how the interpolant framework leads to a natural formulation of the Schr\"odinger bridge problem between two densities.

In Section~\ref{sec:gen}, we discuss a special case in which the interpolant is spatially linear in $x_0$ and $x_1$. 
%
In this case, the velocity field can be factorized, which we show in Section~\ref{sec:factor} leads to a simpler learning problem. 
%
We detail specific choices of linear interpolants in Section~\ref{sec:specific:a:b:c}, and in Section~\ref{sec:impact:gam} we illustrate  how these choices influence the performance of the resulting generative model, with a particular focus on the role of the latent variable and the diffusion coefficient. 
%
For exposition, we focus on Gaussian mixture densities, for which the drift coefficients can be computed analytically. 
%
We provide the resulting formula in Appendix~\ref{app:Gauss:mixt}. 
%
Finally, in Section~\ref{sec:spatil:lin:os}, we discuss the case of spatially linear one-sided interpolants. 
 

In Section~\ref{sec:connection}, we formalize the connection between stochastic interpolants and related classes of generative models.
%
In Section~\ref{sec:SBDM}, we show that score-based diffusion models can be re-written as one-sided interpolants after a reparameterization of time; we highlight how this approach eliminates singularities that appear when naively compressing score-based diffusion onto a finite-time interval.
%
In Section~\ref{sec:denoiser}, we show how interpolants can be used to derive the Bayes-optimal estimator for a denoiser, and we show how this approach can be iterated to create a generative model. 
%
In Section~\ref{sec:rect}, we consider the possibility of rectifying the flow map of a learned generative model.
%
We show that the rectification procedure does not change the underlying generative model, though it may change the time-dependent density of the interpolant.

In Section~\ref{sec:practical}, we provide the details of practical algorithms associated with the mathematical results presented above. 
%
In Section~\ref{sec:learning}, we describe how to numerically estimate the objectives given empirical datasets from the base and the target.
%
In Section~\ref{sec:sampling}, we complement this discussion on \textit{learning} with algorithms for \textit{sampling} with the ODE or an SDE. 

We provide numerical demonstrations in line with these recommendations in Section~\ref{sec:numerics}, and we conclude with some remarks in Section~\ref{sec:conc}.

\subsection{Related work}
\label{sec:related}

\paragraph{Deterministic Transport and Normalizing Flows.}

Transport-based sampling and density estimation has its contemporary roots in Gaussianizing data via maximum entropy methods \citep{Friedman1987,Chen2000, tabak2010, tabak2013}. The change of measure under such transformation is the backbone of normalizing flow models. The first neural network realizations of these methods arose through imposing clever structure on the transformation to make the change of measure tractable in discrete, sequential steps \citep{rezende2015, dinh2017density, papamakarios2017, huang2018, durkan2019}. A continuous time version of this procedure was made possible by viewing the map $T= X_t(x)$ as the solution of an ODE  \citep{chen2018, grathwohl2018scalable}, whose parametric drift defining the transport is learned via maximum likelihood estimation. Training this way is intractable at scale, as computing the gradient of the objective via the adjoint method requires simulating an ODE. Various methods have introduced regularization on the path taken between the two densities to make the ODE solves more efficient \citep{finlay2020,wu2021,tong_trajectorynet_2020}, but the fundamental difficulty remains. We also work in continuous time; however, our approach allows us to learn the drift without simulation of the dynamics, and can be formulated at sample generation time through either deterministic or stochastic transport.

\paragraph{Stochastic Transport and Score-Based Diffusion Models (SBDMs).}
%
Complementary to approaches based on deterministic maps, recent works have realized that connecting a data distribution to a Gaussian density can be viewed as the evolution of an Ornstein-Ulhenbeck (OU) process which gradually degrades samples from the distribution of interest to Gaussian noise~\citep{dickstein2015, ho2020, Song2019, song2021scorebased}. 
%
The OU process specifies a path in the space of probability densities; this path is simple to traverse in the forward direction by addition of noise, and can be reversed if access to the score of the time-dependent density $\nabla\log\rho(t)$ is available.
%
This score can be approximated through solution of a least-squares regression problem~\citep{hyvarinen05a, vincent_connection_2011}, and the target can be sampled by reversing the path once the score has been learned.
%
Interestingly, the resulting forward and backward stochastic processes have an equivalent formulation (at the distribution level) in terms of a deterministic probability flow equation, first noted by~\cite{Bakry1985, OTTO2000361, Kim2010AGO} and then applied in~\cite{maoutsa2020, song2021mle, kingma2021on, boffi2022}. 
%
The probability flow formulation is useful for density estimation and cross-entropy calculations, but it is worth noting that the probability flow and the reverse-time SDE will have densities that differ when using an approximate score.
%
The SBDM framework, as it has been originally presented, has a number of features which are not \textit{a~priori} well motivated, including the dependence on mapping to a normal density, the complicated tuning of the time parameterization and noise scheduling \citep{xiao2022tackling, hoogeboom2023simple}, and the choice of the underlying stochastic dynamics \citep{dockhorn2022score, Karras2022edm}.
%

\paragraph{Stochastic bridges.}
Starting with~\citep{peluchetti2022nondenoising} there has been some recent effort~\citep{liu2022let,liu2023I2SB,somnath2023aligned} to remove the dependence of SBDMs on the OU process via stochastic bridges, which can be used to connect two arbitrary densities in finite time. 
%
As another step in this direction, we observe here that the key idea behind SBDMs -- the bridging of two densities via a time-dependent density whose evolution equation is available -- can be generalized to a much wider class of processes in a straightforward and computationally accessible manner. 
%
This viewpoint highlights the key property that the construction of the bridge between the two densities is decoupled from the process used to sample it.
%construction of the bridge between the two densities, and the definition process introduced afterwards to sample this bridge are separate procedures.}


\paragraph{Stochastic Interpolants, Rectified Flows, and Flow matching.}
Variants of the stochastic interpolant method presented in \cite{albergo2023building} were also presented in \cite{liu2022, lipman2022}. In \cite{liu2022}, a linear interpolant was proposed with a focus on straight paths. This was employed as a step toward rectifying the transport paths \citep{liu2022-ot} through a procedure that improves sampling efficiency but introduces a bias.  In \cite{lipman2022}, the interpolant picture was assembled from the perspective of conditional probability paths connecting to a Gaussian, where a noise convolution was used to improve the learning at the cost of biasing the method. Extensions of \cite{lipman2022} were presented in \cite{tong2023conditional} that generalize the method beyond the Gaussian base density. In the method proposed here, we introduce an unbiased means to incorporate noise into the process, both via the introduction of a latent variable into the stochastic interpolant and the inclusion of a tunable diffusion coefficient in the associated stochastic generative models. We provide theoretical and practical motivation for the presence of these noise terms. 


\paragraph{Optimal Transport and Schrödinger Bridges.}

There is both theoretical and practical interest in minimizing the transport cost of connecting $\rho_0$ and $\rho_1$. In the case of deterministic maps, this is characterized by the optimal transport problem, and in the case of diffusive maps, by the Schr\"odinger Bridge problem \citep{villani2009optimal, chen2021}. Formally, these two problems can be related by viewing the Schr\"odinger Bridge as an entropy-regularized optimal transport.
%
Optimal transport has primarily been employed as a means to regularize flow-based methods by imposing either a path length penalty \citep{zhang2018, wu2021, finlay2020, tong_trajectorynet_2020} or structure on the parameterization itself \citep{huang2021convex, Yang2022}. 
%
A variety of recent works have formulated the Schr\"odinger problem in the context of a learnable diffusion \citep{bortoli2021diffusion, su2023dual, chen2022likelihood}.  
%
For the case of Gaussians, recent work has also identified an analytical solution~\citep{bunne_schrodinger_2022}.
%
In the interpolant framework, \citep{albergo2023building, liu2022, lipman2022, tong2023conditional} all propose optimal transport extensions to the learning procedure. The method proposed in \cite{liu2022, liu2022-ot} allows one to sequentially lower the transport cost through rectification, at the cost of introducing a bias unless the velocity field is perfectly learned. The method proposed in \cite{albergo2023building} is an unbiased framework at the cost of solving an additional optimization problem over the interpolant function. The statement of optimal transport in \cite{lipman2022} only applies to Gaussians, but is shown to be practically useful in experimental demonstrations. 


In the method proposed below, we provide two approaches for optimizing the transport under a stochastic dynamics. Our primary approach, based on the scheme introduced in \cite{albergo2023building}, is presented in Section~\ref{sec:si:schb}. It offers an alternative route to solve the Schr\"odinger bridge problem under the Benamou-Brenier hydrodynamic formulation of transport by maximizing over the interpolant \citep{benamou2000computational}. However, we stress that this additional optimization step is not necessary in practice, as our approach leads to bias-free generative models for any fixed interpolant. In addition, Section~\ref{sec:rect} discusses an unbiased variant of the rectification scheme proposed in \cite{liu2022}.

\paragraph{Convergence bounds.}
Inspired by the successes of score-based diffusion, significant recent research effort has been expended to understand the control that can be obtained on suitable distances between the distribution of the generative model and the target data distribution, such as $\mathsf{KL}$, $W_2$, or $\mathsf{TV}$. 
%
Perhaps the first line of work in this direction is~\cite{song2021mle}, which showed that standard score-based diffusion training techniques bound the likelihood of the resulting SDE model. 
%
Importantly, as we show here, the likelihood of the corresponding probability flow is not bounded in general by this technique, as first highlighted in the context of SBDM by~\cite{lu2022higherorder}.
%
Control for SBDM-based techniques was later quantified more rigorously under the assumption of functional inequalities in a discretized setting by~\cite{holden_score1}, which were removed by~\cite{holden_score2} and~\cite{sinho_score1} via Girsanov-based techniques.
%
Most relevant to the PDE-based methods considered here is~\cite{holden_score3}, which applies similar techniques to our own in the SBDM context to obtain sharp guarantees with minimal assumptions.

\subsection{Notation}
\label{sec:notations}

Throughout, we denote probability density functions as $\rho_0(x)$, $\rho_1(x)$, and $\rho(t,x)$, with $t\in[0,1]$ and $x\in\RR^d$, omitting the function arguments when clear from the context. We proceed similarly for other functions of time and space, such as $b(t,x)$ or $I(t,x_0,x_1)$. We use the subscript $t$ to denote the time-dependency of stochastic processes, such as the stochastic interpolant $x_t$ or the Wiener process $W_t$. To specify that the random variable $x_0$ is drawn from the probability distribution with density $\rho_0$, say, with a slight abuse of notations we use $x_0\sim\rho_0$. Similarly, we use ${\sf N}(0,\Id)$ to denote both the density and the distribution of the Gaussian random variable with mean zero and covariance identity. We denote expectation by $\EE$, and usually specify the random variables this expectation is taken over. With a slight abuse of terminology, we say that the law of the process $x_t$ is $\rho(t)$ if $\rho(t)$ is the density of the probability distribution of $x_t$ at time $t$.

We use standard notation for function spaces: for example, $C^1([0,1])$ is the space of continuously differentiable functions from $[0,1]$ to $\RR$, $(C^2(\RR^d))^d$ is the space of twice continuously differentiable functions from $\RR^d$ to $\RR^d$, and $C^p_0(\RR^d)$ is the space of compactly supported functions from $\RR^d$ to $\RR$ that are continuously differentiable $p$ times. Given a function $b:[0,1]\times \RR^d \to \RR^d$ with value $b(t,x)$ at $(t,x)$, we use $b \in C^1([0,1]; (C^2(\RR^d))^d)$ to indicate that $b$ is continuously differentiable in $t$ for all $(t,x)\in[0,1]\times \RR^d$ and that $b(t,\cdot)$ is an element of $(C^2(\RR^d))^d$ for all $t\in[0,1]$. 
