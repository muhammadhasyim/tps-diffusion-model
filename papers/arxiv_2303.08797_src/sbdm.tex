Score-based diffusion models (SBDM) are based on variants of the Ornstein-Uhlenbeck process
\begin{equation}
    \label{eq:sbdm:sde}
    d Z_\tau = - Z_\tau dt + \sqrt{2} dW_\tau, \qquad Z_{\tau=0} \sim \rho_1,
\end{equation}
which has the property that the marginal density of its solution at time $\tau$ converges to a standard normal as $\tau$ tends towards infinity. By learning the score of the density of $Z_\tau$, we can write the associated backward SDE for~\eqref{eq:sbdm:sde}, which can then be used as a generative model -- this backwards SDE is also the one that is used in the stochastic localization process, see~\cite{montanari2023sampling}.

To see the connection with stochastic interpolants, notice that the solution of \eqref{eq:sbdm:sde} from the initial condition $Z_{\tau=0} = x_1\sim \rho_1$ can be written exactly as
\begin{equation}
    \label{eq:sbdm:sde:sol}
    Z_\tau = x_1 e^{-\tau}  + \sqrt{2} \int_0^\tau e^{-\tau+s} dW_s.
\end{equation}
As a result, the law of $Z_\tau$ conditioned on $Z_{\tau=0} = x_1$ is given by
\begin{equation}
    \label{eq:sbdm:sde:sol:2}
    Z_\tau \sim {\sf N}(x_1 e^{-\tau}, (1-e^{-2\tau})\Id),
\end{equation}
for any time $\tau\in[0,\infty)$. This is also the law of the process
\begin{equation}
    \label{eq:sbdm:sde:sol:3}
    y_\tau = x_1 e^{-\tau} + \sqrt{1-e^{-2\tau}}\, z, \qquad z\sim {\sf N}(0,\Id), \qquad \tau \in [0, \infty).
\end{equation}
If we let $x_1\sim \rho_1$ with $x_1\perp z$, the process $y_\tau$ is similar to a one-sided stochastic interpolant, except 
the density of $y_\tau$ only converges to ${\sf N}(0,\Id)$ as $\tau\to\infty$; by contrast, the one-sided interpolants we introduced in Section~\ref{sec:onesided} converge on the finite interval $[0, 1]$.
%
%
In SBDM, this is handled by capping the evolution of $Z_\tau$ to a finite time interval $[0, T]$ with $T < \infty$, and then by using the backward SDE associated with~\eqref{eq:sbdm:sde} restricted to $[0,T]$. 
%
However, this introduces a bias that is not present with one-sided stochastic interpolants, because the final condition used for the backwards SDE in SBDM is drawn from ${\sf N}(0,\Id)$ even though the density of the process~\eqref{eq:sbdm:sde} is not Gaussian at time~$T$.


We can, however, turn~\eqref{eq:sbdm:sde:sol:3} into a one-sided linear stochastic interpolant by defining $t=e^{-\tau}$ and by choosing $\alpha(t)$ and $\beta(t)$ in~\eqref{eq:interp:os:lin} to have a specific form. More precisely, evaluating~\eqref{eq:sbdm:sde:sol:3} at $\tau = -\log t$,
\begin{equation}
    \label{eq:link:os:sbdm}
    y_{\tau=-\log t} = \sqrt{1-t^2} z + t x_1 \equiv  x^\OSLIN_t   \quad \text{for} \quad  \alpha(t) = \sqrt{1-t^2}, \quad \beta(t) =t.
\end{equation}
With this choice of $\alpha(t)$ and $\beta(t)$, from~\eqref{eq:b:ode:os:lin} we get the velocity field
\begin{equation}
    \label{eq:b:sbdm}
    b_\ODE(t,x) = -\frac{t}{\sqrt{1-t^2}} \eta^\OS_z(t,x) + \eta^\OS_1(t,x) \equiv t s(t,x) + \eta^\OS_1(t,x)
\end{equation}
where $\eta_z^\OS$ and $\eta_1^\OS$ are defined in~\eqref{eq:eta:os}. 
%
This expression shows that the velocity $b_\ODE$ used in the probability flow ODE~\eqref{eq:ode:1} is well-behaved at all times, including at $t=1$ where $\dot\alpha(t)$ is singular. 
%
The same is true for the drift $b_\fwd(t,x) = b_\ODE(t,x) + \eps(t) s(t,x)$ used in the forward SDE~\eqref{eq:sde:1}, regardless of the choice of $\eps \in C^0([0,1])$ with $\eps(t)\ge 0$. 
%
This shows that casting SBDM into a one-sided linear stochastic interpolant~\eqref{eq:interp:os:lin} allows the construction of \textit{unbiased} generative models that operate on $t\in[0,1]$.
%
This comes at no extra computational cost, since only one of the two functions defined in~\eqref{eq:eta:os} needs to be estimated, which is akin to estimating the score in SBDM. 

It is worth comparing the above procedure to an equivalent change of time at the level of the diffusion process~\eqref{eq:sbdm:sde}, which we now show leads to singular terms that pose numerical and analytical difficulties.
%
Indeed, if we define $Z^\rev_t = Z_{\tau=-\log t}$, from~\eqref{eq:sbdm:sde} we obtain
\begin{equation}
\label{eq:sde:sbdm:tchange:r}
d Z^\rev_t = t^{-1} Z^\rev_tdt + \sqrt{2 t^{-1}} dW^\rev_t, \quad Z^\rev_{t=1} \sim \rho_1,
\end{equation}
to be solved backwards in time. 
%
Because of the factor $t^{-1}$, this SDE cannot easily be solved until $t=0$, which corresponds to $\tau=\infty$ in the original~\eqref{eq:sbdm:sde}. 
%
For the same reason, the forward SDE associated with~\eqref{eq:sde:sbdm:tchange:r}
\begin{equation}
    \label{eq:sde:sbdm:tchange:f}
    d Z^\fwd_t = t^{-1} Z^\fwd_tdt + 2 t^{-1}s(t,Z^\fwd_t) dt + \sqrt{2 t^{-1}} dW_t, 
\end{equation}
cannot be solved from $t=0$, where formally $Z^\fwd_{t=0} \stackrel{d}{=} Z^\rev_{t=0} = Z_{\tau=\infty} \sim \mathsf N(0,\Id)$.
%
This means it cannot be used as a generative model unless we start from some $t>0$, which introduces a bias. 
%
Importantly, this problem does not arise with the stochastic interpolant framework, because the construction of the density $\rho(t)$ connecting $\rho_0$ and $\rho_1$ is handled separately from the construction of the process that generates samples from $\rho(t)$.
%
By contrast, SBDM combines these two operations into one, leading to the singularity at $t=0$ in the coefficients in~\eqref{eq:sde:sbdm:tchange:r} and \eqref{eq:sde:sbdm:tchange:f}.

\begin{remark}
    To emphasize the last point made above, we stress that there is no contradiction between having  a singular drift and diffusion coefficient in~\eqref{eq:sde:sbdm:tchange:f}, and being able to write a nonsingular SDE with stochastic interpolants. To see why, notice that the stochastic interpolant tells us that we can change the diffusion coefficient in~\eqref{eq:sde:sbdm:tchange:f} to any nonsingular $\eps\in C^0([0,1])$ with $\eps(t)\ge 0$ and replace this SDE with
\begin{equation}
    \label{eq:sde:sbdm:tchange:f:2}
    d X^\fwd_t = t^{-1} Z^\fwd_tdt + ( t^{-1} +\eps(t)) s(t,X^\fwd_t) dt + \sqrt{2 \eps(t)} dW_t, 
\end{equation}
This SDE has the property that $X^\fwd_{t=1}\sim \rho_1$ if $X^\fwd_{t=0}\sim \rho_0$, and its drift is also nonsingular at $t=0$ and given precisely by~\eqref{eq:b:sbdm}. Indeed, using the constraint~\eqref{eq:eta:os:c}, which here reads $x= \sqrt{1-t^2} \eta^\OS_z(t,x) + t \eta^\OS_1(t,x)\equiv -(1-t^2) s(t,x) + t \eta^\OS_1(t,x)$, it is easy to see that
\begin{equation}
    \label{eq:sde:sbdm:tchange:f:3}
    t^{-1} x +  t^{-1} s(t,x) = t s(t,x) + \eta_1^\OS(t,x), 
\end{equation}
which is nonsingular at $t=0$.
\end{remark}
